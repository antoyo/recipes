% Source: http://www.auxdelicesdupalais.net/recette-maki-saumon-avocat-facile.html

\preparationtime{15}

\cookingtime{15}

\totaltime{}

\portions{6}

\begin{ingredients}
    \item 250 g de riz rond japonais
    \item 3 cuillères à soupe de vinaigre de riz
    \item 1 cuillère à soupe de sucre
    \item 1 cuillère à café de sel
    \item 6 feuilles d’algue pour maki (nori)
    \item 2 avocats
    \item 1 concombre
    \item 200 g de saumon frais ou fumé
\end{ingredients}

\begin{steps}
    \item Laver abondamment le riz à l’eau froide. Bien égoutter. Placer le riz dans une casserole puis ajouter 2 fois le volume de riz en eau. Couvrir la casserole et laisser bouillir pendant 5 minutes en surveillant pour que ça l’eau ne déborde pas. Laisser mijoter à faible température pendant 10 à 12 minutes ou jusqu’à absorption de l’eau. Éteindre le feu et laisser reposer le riz pendant 10 minutes.
    \item Dans un petit bol mettre le vinaigre de riz, le sucre et le sel mélanger bien pour que le sucre se dissous totalement. Verser cette préparation sur le riz puis mélanger. Le riz est prêt.
    \item Couper le saumon en lamelles si il n’est pas déjà prédécoupé.
    \item Couper les avocats en deux, retirer le noyau et couper ces moitiés en 2 pour facilement les extraire. Éplucher et couper les concombres en lamelles les plus fines possible.
    \item Poser sur une feuille de cuisson ou sur une natte à sushi une feuille d’algue sur le bord le plus proche. La face du dessus doit être celle qui est la plus texturée. Humidifier légèrement la zone sur laquelle on va mettre le riz puis en étaler en partant du bord de la feuille. Une fois le riz bien étalé, placer dans le sens de la largeur le poisson et les légumes. Humidifier légèrement la partie opposée de la feuille surtout le bord et rouler immédiatement.
    \item Rouler et appuyer fermement en même temps. Puis pousser le bord vers l’intérieur et appuyer fermement pour tasser l’ensemble.
    \item Répéter l’opération autant de fois que voulu.
    \item À l’aide d’un couteau propre, tranchant et légèrement humide, couper le rouleau en deux et placer les deux extrémités côte à côte.
    \item Découper des tronçons de 2 à 3 cm en effectuant un mouvement rapide comme avec une scie sans appuyer.
    \item Pour plus de goût vous pouvez ajouter aux makis des graines de sésame grillé ou des cacahuètes.
\end{steps}
