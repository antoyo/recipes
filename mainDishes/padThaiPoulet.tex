% Source: http://www.recettes.qc.ca/recettes/recette/pad-thai-au-poulet-1

\preparationtime{10}

\cookingtime{10}

\totaltime{}

\portions{4}

\begin{ingredients}
    \item 8 oz nouilles de riz
    \item 2 poitrines de poulet coupées en petites lanières
    \item sel et poivre
    \item 1 cuillère à soupe huile végétale
    \item 2 oeufs battus
    \item 1 tasse fève germée
    \item 4 oignons verts tranchés grossièrement
    \item 1 grosse échalote française émincée
    \item 1/3 tasse hachée coriandre (optionnel)
    \item arachides hachées (optionnel)
    \item 1/4 tasse ketchup
    \item 2 cuillères à café sauce sriracha (ajouter seulement 1 c. à thé si vous préférez une version moins épicée)
    \item 2 cuillères à soupe sauce de poisson
    \item 4 gousses d'ail émincées
    \item 1 cuillère à soupe beurre d'arachide
    \item 1 cuillère à café sauce soya
    \item jus de lime
\end{ingredients}

\begin{steps}
    \item Cuire les nouilles une à deux minutes de moins qu’indiqué sur l'emballage. Les nouilles doivent être al dente pour éviter qu’elles ne deviennent pâteuses une fois ajoutées au wok.
    \item Préparer la sauce en mélangeant tous les ingrédients. Réserver.
    \item Assaisonner le poulet de sel et poivre. Chauffer l’huile dans un wok à feu moyen-vif et y sauter le poulet jusqu’à ce qu’il soit cuit. Transférer à un bol et réserver.
    \item Cuire ensuite les œufs battus dans le wok. Ajouter les échalotes et les fèves germées. Incorporer le poulet et la sauce.
    \item Ajouter les nouilles et bien mélanger pour enrober de sauce. Servir saupoudrer de coriandre et d’arachides hachées.
\end{steps}
