% Source: http://www.recettes.qc.ca/recettes/recette/cotes-levees-de-flanc-style-memphis

\preparationtime{20}

\cookingtime{240}

\totaltime{}

\portions{4}

\begin{ingredients}
    \item[] \cookingpart{Marinade sèche}
    \item 8 cuillères à table sucre brun
    \item 4 cuillères à table paprika
    \item 6 cuillères à thé sel casher
    \item 4 cuillères à thé poivre noir
    \item 2 cuillères à thé sel d'ail
    \item 2 cuillères à thé sel d'oignon
    \item 2 cuillères à thé sel de céleri
    \item 2 cuillères à thé poivre de Cayenne
    \item 2 cuillères à thé cumin moulu
    \item[] \cookingpart{Sauce rouge style Memphis}
    \item 1 à 1 1/4 tasse ketchup
    \item 1 tasse eau
    \item 1 tasse sirop d'érable
    \item 3/4 tasses vinaigre
    \item 3/4 tasses pâte de tomates
    \item 3/4 tasses sucre brun
    \item 4 cuillères à table miel
    \item 3 cuillères à table mélasse
    \item 4 cuillères à thé sel
    \item 4 cuillères à thé sauce Worcestershire
    \item 1 cuillère à table sauce aux pommes
    \item 1 à 1 1/2 c. à thé sauce soya
    \item 1 cuillère à thé poudre d'oignon
    \item 3/4 cuillères à thé fécule de maïs
    \item 1/2 cuillère à thé moutarde en poudre
    \item 1/2 cuillère à thé piment de Cayenne
    \item 1/2 cuillère à thé poivre noir
    \item 1/8 cuillère à thé cumin moulu
    \item 1/8 cuillère à thé poudre d'ail
    \item 1/8 cuillère à thé poivre blanc ou rose
    \item 1/8 cuillère à thé graine de céleri
    \item[] \cookingpart{Viande}
    \item 4 côtes levées de flanc
\end{ingredients}

\begin{steps}
    \item Pour la marinade sèche : mélangez tous les ingrédients de la marinade sèche dans un contenant avec lequel vous pourrez saupoudrer.
    \item Pour la sauce : mélangez tous les ingrédients de la sauce dans un chaudron. Faites bouillir et réduire en mélangeant le tout. Laissez la sauce refroidir. Conservez dans un pot au réfrigérateur.
    \item Pour les côtes levées : retirez la membrane à l'intérieur des côtes avec un couteau à beurre et un essuie-tout. Saupoudrez généreusement la marinade sèche sur la viande (il ne devrait pas en rester après avoir saupoudré les 4 côtés). Pressez le mélange contre les côtes pour faire pénétrer.
    \item Allumez seulement un côté du barbecue au charbon ou au bois afin d'obtenir une chaleur indirecte de 250°F. Déposez les côtes levées le plus loin possible de la chaleur dans le barbecue. Fermez le couvercle et laissez ainsi 4 heures, ou jusqu'à ce que les côtes levées soient tendres. La chaleur interne doit atteindre 190 à 195°F.
    \item Faites réchauffer la sauce et badigeonnez-en généreusement les côtes. Grillez la viande à feu moyen environ 5 minutes ou jusqu'à ce que la sauce soit bien caramélisée.
    \item Coupez les côtes en frottant l'os pour conserver le maximum de viande sur les os. Gardez 2 os sur 3 et conservez le reste. On appelle cette coupe "Hollywood".
\end{steps}
