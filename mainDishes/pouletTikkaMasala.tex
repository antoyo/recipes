% Source: https://www.ricardocuisine.com/recettes/3849-poulet-tikka-masala

\preparationtime{20}

\cookingtime{35}

\totaltime{}

\portions{4}

\begin{ingredients}
    \item[] \cookingpart{Marinade sèche}
    \item 15 ml (1 c. à soupe) de garam masala
    \item 15 ml (1 c. à soupe) de poudre de chili
    \item 5 ml (1 c. à thé) de cumin moulu
    \item 5 ml (1 c. à thé) de curcuma moulu
    \item 1 ml (1/4 c. à thé) de cardamome moulue
    \item[] \cookingpart{Poulet}
    \item 675 g (1 1/2 lb) de hauts de cuisses de poulet désossés et sans la peau, coupés en cubes (ou poitrines)
    \item 125 ml (1/2 tasse) de noix de cajou grillées
    \item 45 ml (3 c. à soupe) d'huile d'olive
    \item 1 oignon, haché finement
    \item 1 piment jalapeno, épépiné et haché finement
    \item 2 gousses d'ail, hachées finement
    \item 5 ml (1 c. à thé) de gingembre frais haché finement
    \item 30 ml (2 c. à soupe) de pâte de tomates ou au goût
    \item 1 grosse tomate, épépinée et coupée en dés
    \item 125 ml (1/2 tasse) de crème 35\%
    \item 125 ml (1/2 tasse) de yogourt nature 10\%
    \item Sel et poivre
\end{ingredients}

\begin{steps}
    \item[] \cookingpart{Marinade sèche}
    \item Dans un grand bol, mélanger tous les ingrédients. Ajouter la viande et bien l'enrober du mélange d'épices. Réserver.
    \item[] \cookingpart{Poulet}
    \item Dans un petit robot culinaire ou dans un moulin à café, hacher finement les noix de cajou. Réserver.
    \item Dans une grande poêle ou une grande casserole, dorer le poulet dans l'huile. Saler et poivrer. Réserver sur une assiette.
    \item Dans la même poêle, dorer l'oignon avec le piment. Ajouter de l'huile au besoin. Ajouter l'ail, le gingembre et la pâte de tomates. Poursuivre la cuisson 1 minute. Remettre le poulet dans la poêle. Ajouter les noix de cajou et le reste des ingrédients. Porter à ébullition. Couvrir et laisser mijoter doucement environ 20 minutes en remuant fréquemment.
    \item Servir sur un riz basmati.
\end{steps}
