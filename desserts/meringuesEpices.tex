% Source: https://www.ricardocuisine.com/recettes/6537-meringues-aux-epices

\preparationtime{25}

\cookingtime{180}

\totaltime{}

\portions{120}

\begin{ingredients}
    \item 10 ml (2 c. à thé) de fécule de maïs
    \item 5 ml (1 c. à thé) de vinaigre blanc
    \item 2,5 ml (1/2 c. à thé) d’eau
    \item 1 pincée de cannelle moulue
    \item 1 pincée de gingembre moulu
    \item 1 pincée de muscade moulue
    \item 4 blancs d’oeufs, à la température ambiante
    \item 250 ml (1 tasse) de sucre
    \item 115 g (4 oz) de chocolat noir, fondu et tempéré (facultatif)
\end{ingredients}

\begin{steps}
    \item Placer deux grilles au centre du four. Préchauffer le four à 80 °C (170 °F). Tapisser deux plaques à biscuits de papier parchemin.
    \item Dans un petit bol, mélanger la fécule, le vinaigre, l’eau et les épices. Réserver.
    \item Dans un bol, fouetter les blancs d’oeufs au batteur électrique jusqu’à ce qu’ils forment des pics mous. Ajouter le sucre en fine pluie en continuant de fouetter jusqu’à l’obtention de pics fermes. À la spatule, incorporer le mélange d’épices en pliant délicatement.
    \item Transvider la meringue dans une poche à pâtisserie munie d’une douille ronde unie. Façonner de petites meringues d’environ 15 ml (1 c. à soupe) et les répartir sur les plaques.
    \item Cuire au four 3 heures ou jusqu’à ce qu’elles soient sèches et se décollent du papier, en faisant pivoter les plaques sur elles-mêmes et en les changeant de grille à mi-cuisson. Laisser refroidir complètement sur les plaques.
    \item Tremper la base des meringues dans le chocolat (voir note). Les retourner sur les plaques pour laisser figer le chocolat. Les meringues se conservent environ 3 semaines dans un contenant hermétique, dans un endroit frais et sec.
    \item Note: Pour que les meringues restent craquantes plus longtemps, trempez-les complètement dans le chocolat.
\end{steps}
