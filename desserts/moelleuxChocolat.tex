%
%   Copyright (c) 2015 Boucher, Antoni <bouanto@gmail.com>
%   
%   Permission is hereby granted, free of charge, to any person obtaining a copy of
%   this software and associated documentation files (the "Software"), to deal in
%   the Software without restriction, including without limitation the rights to
%   use, copy, modify, merge, publish, distribute, sublicense, and/or sell copies of
%   the Software, and to permit persons to whom the Software is furnished to do so,
%   subject to the following conditions:
%   
%   The above copyright notice and this permission notice shall be included in all
%   copies or substantial portions of the Software.
%   
%   THE SOFTWARE IS PROVIDED "AS IS", WITHOUT WARRANTY OF ANY KIND, EXPRESS OR
%   IMPLIED, INCLUDING BUT NOT LIMITED TO THE WARRANTIES OF MERCHANTABILITY, FITNESS
%   FOR A PARTICULAR PURPOSE AND NONINFRINGEMENT. IN NO EVENT SHALL THE AUTHORS OR
%   COPYRIGHT HOLDERS BE LIABLE FOR ANY CLAIM, DAMAGES OR OTHER LIABILITY, WHETHER
%   IN AN ACTION OF CONTRACT, TORT OR OTHERWISE, ARISING FROM, OUT OF OR IN
%   CONNECTION WITH THE SOFTWARE OR THE USE OR OTHER DEALINGS IN THE SOFTWARE.
%

% Source: https://zestedesavoir.com/articles/219/interview-rencontre-avec-taguan/

\preparationtime{15}

\cookingtime{12}

\totaltime{}

\portions{12}

\begin{ingredients}
    \item 3 oeufs
    \item 1/2 tasse de farine
    \item 1/3 tasse de sucre
    \item 120 g chocolat noir (idéalement du chocolat patissier/culinaire)
    \item 1/2 tasse de beurre
    \item 12 carrés de chocolat blanc
\end{ingredients}

\begin{steps}
    \item Préchauffer le four à 350°F (180°C).
    \item Dans un saladier en inox, mélangez les oeufs, le sucre et la farine au fouet.
    \item Faites fondre le beurre et le chocolat noir au bain-marie, ou sur feu très très doux. Ajoutez ce mélange dans le saladier et mélangez à nouveau au fouet.
    \item Remplissez vos empreintes jusqu'à environ la moitié et déposez un carré de chocolat blanc dans chaque empreinte (sans l'enfoncer). Remettez un peu d'appareil par dessus le chocolat blanc (remplissez aux 2/3, maximum 3/4, ça va gonfler à la cuisson).
    \item Faites cuire vos moelleux 12 minutes à 180°C, option chaleur tournante si possible. Laissez refroidir quelques minutes avant de démouler et dégustez sans trop trainer pour que le chocolat blanc soit bien chaud et fondu à l'intérieur !
\end{steps}
