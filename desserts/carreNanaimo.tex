%
%   Copyright (c) 2014 Boucher, Antoni <bouanto@gmail.com>
%   
%   Permission is hereby granted, free of charge, to any person obtaining a copy of
%   this software and associated documentation files (the "Software"), to deal in
%   the Software without restriction, including without limitation the rights to
%   use, copy, modify, merge, publish, distribute, sublicense, and/or sell copies of
%   the Software, and to permit persons to whom the Software is furnished to do so,
%   subject to the following conditions:
%   
%   The above copyright notice and this permission notice shall be included in all
%   copies or substantial portions of the Software.
%   
%   THE SOFTWARE IS PROVIDED "AS IS", WITHOUT WARRANTY OF ANY KIND, EXPRESS OR
%   IMPLIED, INCLUDING BUT NOT LIMITED TO THE WARRANTIES OF MERCHANTABILITY, FITNESS
%   FOR A PARTICULAR PURPOSE AND NONINFRINGEMENT. IN NO EVENT SHALL THE AUTHORS OR
%   COPYRIGHT HOLDERS BE LIABLE FOR ANY CLAIM, DAMAGES OR OTHER LIABILITY, WHETHER
%   IN AN ACTION OF CONTRACT, TORT OR OTHERWISE, ARISING FROM, OUT OF OR IN
%   CONNECTION WITH THE SOFTWARE OR THE USE OR OTHER DEALINGS IN THE SOFTWARE.
%

% Source: http://www.recettes.qc.ca/recette/carre-domino-ou-nanaimo-214756

\preparationtime{30}

\macerationtime[3]{0}

\totaltime{}

\portions{15}

\begin{ingredients}
    \item[] \cookingpart{Base}
    \item 1/2 tasse beurre ramolli
    \item 2 carrés de chocolat mi-sucré (de type Baker's) fondus
    \item 1 c. à thé extrait de vanille pure
    \item 2 tasses chapelure de biscuits Graham
    \item 1 tasse noix de coco non sucrée en flocons
    \item 1/2 tasse noix de Grenoble hachées
    \item Eau
    \\
    \item[] \cookingpart{Garniture}
    \item 1/2 tasse beurre
    \item 4 c. à table lait ou crème 15 \%
    \item 3 c. à table poudre pour pouding à la vanille
    \item 2 1/2 tasse sucre à glacer ou plus si nécessaire
    \\
    \item[] \cookingpart{Glace}
    \item 5 carrés de chocolat mi-sucré (de type Baker's)
    \item 1 c. à table beurre
\end{ingredients}

\begin{steps}
    \item Tapisser le fond et les côtés d'un moule rectangulaire (environ 10 x 15 pouces) de pellicule plastique en laissant dépasser tout le tour du moule. Réserver.
    \item Faire fondre les deux carrés de chocolat au micro-ondes avec un peu d'eau dans un grand bol, puis ajouter le beurre et la vanille au chocolat. Mélanger ensemble.
    \item Ajouter la chapelure, la noix de coco et les noix de Grenoble. Mélanger à la cuillère de bois et étendre uniformément dans un moule vaporisé ou graissé (10 × 15 pouces). Réfrigérer.
    \item Pour la garniture, réduire en crème le beurre, le lait et la poudre à pouding.
    \item Ajouter graduellement le sucre en battant bien après chaque addition. En ajouter plus si nécessaire, jusqu'à ce que la garniture soit bien épaisse. Étaler sur la base et réfrigérer au moins 15 minutes.
    \item Pour le dessus, dans un bol allant au micro-ondes, faire fondre à intensité moyenne le chocolat et le beurre pendant 2 à 3 minutes, jusqu'à ce que le beurre soit fondu.
    \item Remuer à la cuillère jusqu'à ce que le chocolat soit complètement fondu.
    \item Étaler sur la garniture. Tracer les lignes pendant que le chocolat est encore chaud et réfrigérer pendant 3 heures.
    \item Démouler en tirant sur la pellicule plastique, placer sur le plan de travail et couper à l'aide d'un bon couteau sur les lignes déjà tracées.
\end{steps}
