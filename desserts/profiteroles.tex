%
%   Copyright (c) 2014 Boucher, Antoni <bouanto@gmail.com>
%   
%   Permission is hereby granted, free of charge, to any person obtaining a copy of
%   this software and associated documentation files (the "Software"), to deal in
%   the Software without restriction, including without limitation the rights to
%   use, copy, modify, merge, publish, distribute, sublicense, and/or sell copies of
%   the Software, and to permit persons to whom the Software is furnished to do so,
%   subject to the following conditions:
%   
%   The above copyright notice and this permission notice shall be included in all
%   copies or substantial portions of the Software.
%   
%   THE SOFTWARE IS PROVIDED "AS IS", WITHOUT WARRANTY OF ANY KIND, EXPRESS OR
%   IMPLIED, INCLUDING BUT NOT LIMITED TO THE WARRANTIES OF MERCHANTABILITY, FITNESS
%   FOR A PARTICULAR PURPOSE AND NONINFRINGEMENT. IN NO EVENT SHALL THE AUTHORS OR
%   COPYRIGHT HOLDERS BE LIABLE FOR ANY CLAIM, DAMAGES OR OTHER LIABILITY, WHETHER
%   IN AN ACTION OF CONTRACT, TORT OR OTHERWISE, ARISING FROM, OUT OF OR IN
%   CONNECTION WITH THE SOFTWARE OR THE USE OR OTHER DEALINGS IN THE SOFTWARE.
%

% Source: http://www.recettes.qc.ca/recette/profiteroles-avec-christophe-morel-4257

\preparationtime{35}

\cookingtime{25}

\totaltime{}

\portions{8}

\begin{ingredients}
    \item[] \cookingpart{Profiteroles}
    \item 3/4 tasse farine
    \item 1/4 tasse beurre
    \item 1/3 tasse lait
    \item 1/3 tasse eau
    \item 1 pincée sucre blanc granulé
    \item 3 œufs
    \item 1 pincée sel
    \\
    \item[] \cookingpart{Sauce au chocolat}
    \item 200 g chocolat
    \item 1 tasse crème
\end{ingredients}

\begin{steps}
    \item[] \cookingpart{Choux}
    \item Allumer votre four à 400°F (200°C).
    \item Amener l'eau à ébullition dans une casserole et ajouter le beurre, le sel et le sucre.
    \item Ajouter la farine d'un seul coup. Brasser rapidement jusqu'à ce qu'une boule se forme.
    \item Faire cuire la pâte jusqu'à ce qu'elle n'adhère plus à la cuillère ou au bout des doigts lorsqu'on la pince. Mélanger constamment durant la cuisson.
    \item Retirer la casserole du feu et verser la pâte dans un bol. Laisser refroidir 3 minutes.
    \item Incorporer les œufs un à un, tout en mélangeant vigoureusement après chaque addition. La pâte obtenue doit être souple et brillante.
    \item Déposer la pâte en forme de choux sur un papier parchemin ou papier en silicone. Enfourner et éteigner le four pour 10min. Ensuite, allumer le four a 320°F (165°C) et continuer à cuire pour 15 min.
    \\
    \item[] \cookingpart{Sauce}
    \item Réchauffer la crème. Ajouter la crème chaude sur le chocolat et lier à l'aide d'un mélangeur.
\end{steps}
