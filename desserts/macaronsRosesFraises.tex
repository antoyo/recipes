% Source: https://www.ricardocuisine.com/recettes/5297-macarons-roses-aux-fraises

\preparationtime{20}

\cookingtime{12}

\totaltime{}

\portions{20}

\begin{ingredients}
    \item[] \cookingpart{Macarons}
    \item 375 ml (1 ½ tasse) de poudre d’amandes
    \item 250 ml (1 tasse) de sucre à glacer
    \item 3 blancs d’oeufs
    \item Colorant rouge liquide (pas en gel) 8 gouttes pour des macarons rose pâle ou 24 pour des macarons rose foncé
    \item 125 ml (½ tasse) de sucre
    \\
    \item[] \cookingpart{Confiture de fraises}
    \item 500 ml (2 tasses) de fraises coupées en quartiers
    \item 250 ml (1 tasse) de sucre
    \item 15 ml (1 c. à soupe) de jus de citron
\end{ingredients}

\begin{steps}
    \item[] \cookingpart{Macarons}
    \item Placer la grille au centre du four. Préchauffer le four à 180 °C (350 °F). Tapisser deux plaques à biscuits de papier parchemin.
    \item Au robot culinaire, réduire la poudre d’amandes et le sucre à glacer en poudre fine, soit de 2 à 3 minutes. Réserver.
    \item Dans un bol, fouetter les blancs d’oeufs avec le colorant au batteur électrique jusqu’à la formation de pics mous. Ajouter le sucre graduellement en fouettant constamment jusqu’à l’obtention de pics fermes. À l’aide d’une spatule, incorporer vivement le mélange de poudre d’amandes et mélanger jusqu’à ce que la meringue tombe et ramollisse légèrement.
    \item À l’aide d’une poche à pâtisserie munie d’une douille unie d’environ 7 mm (1/2 po) de diamètre, façonner des macarons de 2,5 cm (1 po) de diamètre en les espaçant d’environ 5 cm (2 po). (La pâte devrait s’étendre légèrement. Sinon, c’est qu’elle n’est pas assez ramollie. Sortir de la poche et mélanger de nouveau). Frapper la plaque à plat sur un plan de travail à deux ou trois reprises pour étaler légèrement les macarons.
    \item Cuire au four, une plaque à la fois, environ 12 minutes en laissant la porte entrouverte à l’aide d’une cuillère de bois. Les macarons ne doivent pas dorer. Laisser refroidir complètement sur la plaque. (Pour de plus petits macarons, la cuisson varie entre 8 et 10 minutes. Il faut donc les cuire sur deux plaques différentes.)
    \\
    \item[] \cookingpart{Confiture de fraises}
    \item Dans une casserole, mettre tous les ingrédients. Porter à ébullition en remuant.
    \item Fixer un thermomètre à bonbons au centre de la casserole et laisser mijoter jusqu’à ce que le thermomètre indique 104 °C (219 °F). Écumer soigneusement et remuer fréquemment pendant la cuisson. Laisser refroidir complètement.
    \item Placer la confiture refroidie dans un tamis et laisser égoutter 5 minutes.
    \item Garnir l’intérieur des macarons d’un peu de confiture et refermer en sandwich. On les garnit au dernier moment, car les macarons à la confiture ramollissent rapidement. Il faut les manger quand ils s’assouplissent, mais l’extérieur doit rester craquant.
\end{steps}
